\section{Verifica}

Per verificare che i codici aderiscono agli standard, le componenti che lo compongono non contengano errori e
che facciano effettivamente il loro dovere; si è prestato in uso di diversi strumenti di verifica.

\subsection{W3C HTML Validator}
Lo strumento di validazione di W3C permette di validare il codice HTML, anche quello prodotto da
script PHP visto che `e possibile validare semplicemente facendo un copia-incolla del codice.

\subsection{WAVE Evaluation Tool}
Questo strumento `e stato preso in considerazione fin da subito, infatti ha permesso di scovare problemi
legati all’accessibilit`a. Inoltre il suo utilizzo `e molto semplice, perch´e oltre al sito il servizio pu`o
essere fruibile anche tramite estensioni installabili sui browser pi`u comunemente utilizzati (es. Mozilla
Firefox).

\subsection{W3C CSS Validator}
Questo strumento `e stato utilizzato per validare il codice CSS, il quale deve rispettare rigorosamente
lo standard.

\subsection{PhpCodeChecker}
Lo strumento è stato utilizzato per validare la sintassi del codice PHP.

\subsection{Esprima}
Questo strumento `e servito a controllare la sintassi degli script JavaScript.

\subsection{contrast-ratio.com}
Il servizio offerto da questo sito permette di determinare il livello di contrasto tra due colori, di cui
uno di background e l’altro del font. Ha consentito al gruppo di scegliere e verificare la
combinazione dei colori.
