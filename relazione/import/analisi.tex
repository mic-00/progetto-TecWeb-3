\section{Analisi}

\subsection{Analisi dell'utenza finale}
\textit{AICrash} rappresenta una piattaforma come seconda via per richieste di assistenza.
Si suppone che gli utenti i quali decidono di utilizzeranno questa piattaforma subentrano nelle seguenti situazioni:
\begin{itemize}
	\item Coloro che sono di fretta e non hanno tempo per aspettare la fila;
	\item Coloro che al momento sono impossibilitati a raggiungere il negozio;
	\item Coloro che semplicemente non hanno voglia di fare la fila.
\end{itemize}

Quindi si pensa che  l'utenza finale abbia una buona dimistichezza con i dispositivi tecnologici
in quanto ha deciso di utilizzare gli strumenti via web.

...

\subsection{Motori di ricerca}
Viene elencato di seguito le parole chiave che il sito deve ripondere presentadosi
tra la lista della ricerca.

Una parola chiave è sicuramente \textit{AICrash} o anche altre parole che contengono tale parola, in quanto
e' il nome del sito. Questa ricerca verra' fatta da coloro che hanno gia' visitato o hanno sentito parlare da qualcuno.

Altre parole saranno sicuramente legate all'assistenza dei dispositivi telefonici come \textit{riparazione cellulare}, \textit{guasto cellulare},
\textit{assistenza} ecc..
Questo tipo di ricerca probabilemte sara' eseguito da coloro che vogliono sapere un centro di assistenza vicina a casa loro.

In quanto il sito offre anche il servizio di vendita dei dispositivi ricondizionati, le parole chiavi possono del tipo
\textit{vendita cellulari}, \textit{cellulari ricondizionati}.

Per ultimo, pensando che tra gli utenti ci sono anche quelli piu' esperti, ci saranno casi in cui
venga fatta una ricerca mirata ad un certo brand o modello; quindi tra le set di parole possono comprendere
anche quelle che delle marche e modelli dei cellulari.

\subsection{Tipi di utenti}
\begin{itemize}
	\item \textbf{Utente generico}: si definisce un utente generico quando non registrato o
	un utente registrato ma non ha effettuato il login. Un utente generico non ha accesso 
	al gran parte dei servizi. Tuttavia e' in grado di: 
	\begin{itemize}
	\item Visualizzare le pagine Home, Chi siamo, Negozio e Riparazione;
	\item Sfogliare i prodotti in vendita e visualizzare in dettaglio le informazioni del prodotto;
	\item Registrarsi e eseguire un Login.
	\end{itemize}

	\item \textbf{Utente loggato}: ha accesso a tutti i servizi di un utente generico e inoltre puo':
	\begin{itemize}
	\item Aggiungere un articolo nel carello;
	\item Visualizzare e rimuovere un articolo nella pagina Carello;
	\item Inviare una richiesta di assistenza nella pagina Riparazione;
	\item Ha accesso all'area personale;
	\item Visualizzare le informazioni personali, modificarli o eliminare il proprio account;
	\item Visualizzare le proprie richieste di riparazioni e gli acquisti fatti.
	\end{itemize}

	\item \textbf{Admin}: amministratore puo':
	\begin{itemize}
	\item Visualizzare le richieste di assistenza e risponderle;
	\item Aggiungere o rimuovere un articolo in vendita;
	\item Modificare le informazioni di un articolo.
	\end{itemize}
\end{itemize}



