\section{Descrizione pagine}

\subsection{Home}

La home è la pagina principale del sito; ha scopo di fornire informazionii relativi del sito
all'utente in modo semplice e sintetico.

Il metodo scelto è mettere in viasualizzazione una serie di immagini con ognuna una breve
descrizione relativi ai servizi che fornisce il sito. Le immagini sono applicati come uno slideshow,
viene visualizzato un immagine per volta e con le frecce ai lati dell pagine si puo visualizzare l'immagine
successivo o precedente.

\subsection{Chi siamo}

E' la pagina di rappresentazione ufficiale del sito, oltre alle informazioni su come raggiungere
il negozio, metodo di contatto e orari di apertura, contiene inoltre una breve guida su come 
usufruire i nostri servizi via sito web.

\subsection{Riparazione}

Fornisce il servizio principale del sito, ossia quella di mandare una richiesta di assistenza.

Se l'utente ha eseguito il login, puo' compilare e inviare il form inserendo le informazioni essenziali riguardandi
al danno del dispositivo. Una volta inviata, il negozio emettera' un preventivo sul costo il quale sara' visibile nell'area
utente.

\subsection{Acquisto}

E' la pagina dove sono esposti in vetrina i prodotti in vendita del negozio. L'utente puo' 
sfogliare tra una vasta scelta di brand e modelli e aggiungere un prodotto nel carrello.

\subsection{Carello}

Sono visualizzati tutti i prodotti selezionati dall'utente per l'acquisto. Tale pagina puo' essere
visualizzata eseguendo il login.

\subsection{Login e Registrazione}

Come dal nome, permette all'utente di eseguire il login o una registrazione in caso che l'utente 
non abbia un account. Una volta eseguito il login, verra' visualiizzata l'area utente.

\subsection{Area Utenti}

L'area utente e' caratterizzato daa tre diverse sezioni: 

\begin{itemize}
	\item \textit{Informazioni Personali} e' la sezione dove l'utente puo' visualizzare i propri dati,
	modificarli o eliminare l'account;
	\item\textit{Riparazioni} L'utente puo' visualizzare le proprie richieste di riparazioni, compresi
	quelli in corso e non;
	\item \textit{Acquisti } L'utente puo' visualizzare i propri acquisti
\end{itemize}

\subsection{Area Amministratore}

E' la pagina dedicata all'amministratore e come nell'area utenti e' divisa in diverse sezioni:

\begin{itemize}
	\item \textit{Informazioni Personali} e' la sezione per visualizzare i dati dell'account;
	\item \textit{Gestione modelli} 
	\item \textit{Gestione riparazionli}
	\item \textit{Aggiungi articolo}

\end{itemize}